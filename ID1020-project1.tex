\documentclass[11pt]{article}
\usepackage{geometry}                % See geometry.pdf to learn the layout options. There are lots.
\geometry{a4paper}                   % ... or a4paper or a5paper or ... 
%\geometry{landscape}                % Activate for for rotated page geometry
%\usepackage[parfill]{parskip}    % Activate to begin paragraphs with an empty line rather than an indent
\usepackage{graphicx}



%%Fonts
\usepackage[english]{babel}
\usepackage[T1]{fontenc}
\usepackage{lmodern}

%% Maths
\usepackage{amsmath}
\usepackage{amssymb}
\usepackage{amsthm}
\usepackage{MnSymbol}
\usepackage{mathrsfs}
\usepackage{mathtools}
\usepackage{epstopdf}

%% Listings
\usepackage{listings}
\lstset{frame=tb, basicstyle=\ttfamily, language=Java, commentstyle=\color{red},
  keywordstyle=\color{blue}, numberstyle=\footnotesize,
basicstyle=\footnotesize}
\lstset{breaklines=true, breakatwhitespace=false}
\lstset{numbers=left, numberstyle=\scriptsize, firstnumber=1, columns=fullflexible, showstringspaces=false}

%% Other
\usepackage{enumerate}
\usepackage{paralist}
\usepackage{algpseudocode}
\usepackage{algorithm}
\usepackage{tikz}
\usetikzlibrary{arrows}

\tikzset{
  treenode/.style = {align=center, inner sep=0pt, text centered,
    font=\sffamily},
  arn_n/.style = {treenode, circle, black, font=\sffamily, draw=black,
    fill=none, text width=1.5em}
}

%% Links
\usepackage{color}
\usepackage[pdfusetitle,pdftex,colorlinks]{hyperref}
\hypersetup{
    pdftitle={IS1350 Lab 1},
    pdfauthor={lkroll},
    pdfsubject={Operativsystem IS1350 Lab 1},
    pdfkeywords={},
    bookmarksnumbered=true,     
    bookmarksopen=true,         
    bookmarksopenlevel=1,       
    colorlinks=true,            
    pdfstartview=Fit,           
    pdfpagemode=UseOutlines,    % this is the option you were lookin for
    pdfpagelayout=TwoPageRight,
    urlcolor=blue
}

\DeclareGraphicsRule{.tif}{png}{.png}{`convert #1 `dirname #1`/`basename #1 .tif`.png}

\newcommand {\N} {\mathbb{N}}
\newcommand {\Z} {\mathbb{Z}}
\newcommand {\R} {\mathbb{R}}
\newcommand {\C} {\mathbb{C}}
\newcommand {\Q} {\mathbb{Q}}
\newcommand {\Cx} {\mathcal{C}}
\newcommand {\strcon} {\squigarrowleftright}

\newcommand {\setofmaps} [2] {#2^#1}

\DeclareMathOperator{\Grad}{deg}
\DeclareMathOperator{\mini}{min}
\DeclareMathOperator{\cons}{cons}
\DeclareMathOperator{\nil}{nil}
\DeclareMathOperator{\true}{true}
\DeclareMathOperator{\false}{false}

\title{ID1020 -- Algorithms and Data Structures \\ Project 1 -- VT14 P2}
\author{}
\date{}                                           % Activate to display a given date or no date

\begin{document}
\maketitle
\section{Organisation}
The project is a programming task that is somewhat more involved than the lab. The results will be presented orally and a grade assigned based on the quality of the solution and understanding of the involved concepts.\\

The project itself is split into two parts, \emph{Project 1 (P1)} and \emph{Project 2 (P2)}, such that P1 forms the basis and P2 expands and improves on P1.

\subsection{Dates}
\begin{description}
\item[Submission] Wednesday, Dec. 18th, 23:59 in Bilda
\item[Presentation] Thursday, Dec. 19th, time-slots will be assigned.
\end{description}

\subsection{Goals}
The project has the following goals:
\begin{itemize}
\item Work with the algorithms and data structures presented in the course
\item Reason about usage patterns in a software system and leverage this to make implementation decisions
\item Work on a real problem within the context of the course
\end{itemize}

\subsection{Requirements}
\label{ssec:reqs}
For the project you will need the following:
\begin{itemize}
\item Java
\item Maven
\end{itemize}
cf. Lab 2 for details.

\subsection{Time}
This depends heavily on your experience in writing code for more involved projects.\\
We can only recommend to start early, to get a feeling on how much time you will need to invest.\\
Make sure to think through the problem first before you throw yourself head first into coding. A good design from the beginning can save you many hours of wading through convoluted code later. Especially since you will have to reuse parts of your solution for P2.

\emph{Note: We would appreciate it if you could note the approximate time it took you to do the tasks on your report, so we can improve the estimations in the future.}

\subsection{Notations \& Definitions}
\label{ssec:defs}
We denote the set of all natural numbers with $\N = \{1, 2, 3, 4, \ldots\}$ and $\N_0 = \{0\}\cup\N$. Similarly $\R$ denotes the set of all real numbers and $\R^+$ the set of all positive real numbers.\\
The set of common complexity classes is denoted as follows:
\begin{align*}
\Cx = \left\{f:\N\to\R \middle\mid f(n) \mapsto \left\{\begin{array}{l}1 \\ \log{n} \\ n \\ n\log{n} \\ n^r\mbox{ for some } r\in\R^+ \\ r^n \mbox{ for some } r\in\R^+ \\ n! \end{array}\right.\right\}
\end{align*}
Let $\setofmaps{\N}{\R}$ refer to the set of all functions $f:\N \to \R$ then clearly $\Cx \subseteq \setofmaps{\N}{\R}$.

\section{Background}

\section{Tasks}



\end{document}  